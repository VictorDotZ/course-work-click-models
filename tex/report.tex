\documentclass[diploma]{nanolab2015}

\usepackage{float}
\usepackage{subfigure}
\usepackage{booktabs}
\usepackage{dcolumn}
\usepackage[flushleft]{threeparttable}
\usepackage{makecell}
\usepackage{enumitem}
\usepackage{xparse}
\usepackage{svg}
\newcounter{descriptcount}
\NewDocumentEnvironment{enumdescript}{O{}}{%
    \setcounter{descriptcount}{0}%
    \renewcommand*\descriptionlabel[1]{%
      \stepcounter{descriptcount}%
      \normalfont\bfseries ##1%
    }%
    \description%
  }%
  {\enddescription}



\DeclareMathOperator{\Attention}{Attention}
\DeclareMathOperator{\softmax}{softmax}
\DeclareMathOperator{\ReLU}{ReLU}
\DeclareMathOperator{\MSE}{MSE}
\DeclareMathOperator{\AdamW}{AdamW}


\begin{document}
\begin{titlepage}
    \begin{center}
        \large
        Федеральное государственное бюджетное образовательное учреждение
        высшего образования «Московский государственный университет имени
        М.В.Ломоносова»

        МЕХАНИКО-МАТЕМАТИЧЕСКИЙ ФАКУЛЬТЕТ

        \textbf{Кафедра Математической теории интеллектуальных систем}\\
        \vspace{4cm}
        \textsc{\Large Курсовая работа}\\[5mm]
        {\LARGE Исследование нейросетевых методов построения кликовой модели для задач информационного поиска}
    \end{center}
    \vspace{3cm}
    \null

    \begin{flushright}
        \normalsize \underline{Выполнил:}
        \\студент 531 группы
        \\Зенин В. О.
        \\ \underline{\hspace{4cm}}
    \end{flushright}
    \vspace{1cm}

    \begin{flushright}
        \normalsize \underline{Научный руководитель:}
        \\к.ф.-м.н., н.с Половников В. С.
        \\ \underline{\hspace{4cm}}
    \end{flushright}

    \vfill
    \begin{center}
        \textbf{Москва - 2024}
    \end{center}
\end{titlepage}
\setcounter{page}{3}
\clearpage
\tableofcontents{}  % оглавление
\clearpage
\chapter{Введение}
Информационные системы занимают центральное место в обработке больших объемов данных. С увеличением количества информации важно разрабатывать методы поиска, которые обеспечат пользователям быстрое нахождение нужной информации. Одним из важных источников данных являются действия пользователей при поиске и работе с найденной информацией. Исследуя подобные данные можно извлекать из них различные признаки, способные улучшать качество поисковой системы и лучше удовлетворять потребности пользователя.

Одним из способов работы с такими данными являются кликовые модели, которые анализируют имеющуюся историю взаимодействия пользователей с результатами поисковой выдачи и помогают лучше понять, какие из показанных документов были более или менее полезны. Традиционные подходы к построению кликовых моделей включают в себя различные вероятностные методы, однако с развитием технологий машинного обучения все более актуальным становятся нейросетевые методы, обладающие в некоторых задачах высокой обобщающей способностью, позволяющей эффективно обрабатывать и анализировать сложные и многомерные данные.

Актуальность данной темы обусловлена как практическими потребностями, так и научным интересом. С одной стороны, улучшение алгоритмов информационного поиска напрямую влияет на качество пользовательского опыта и способность поисковых систем решать свою главную задачу по нахождению релевантного контента. С другой стороны, исследование нейросетевых методов в контексте кликовых моделей открывает новые горизонты в области машинного обучения и обработки больших данных.

Целью данной работы является изучение современных нейросетевых методов построения кликовых моделей и их применения в задачах информационного поиска. В работе будут рассмотрены как теоретические аспекты разработки таких моделей, так и практические примеры их внедрения и оценки эффективности. Мы проанализируем существующие подходы, выявим их преимущества и недостатки, а также предложим возможные направления для дальнейших исследований и улучшений.
\newpage
\section{Основные понятия и терминология}
\subsection{Поисковая выдача}

Поисковая выдача — это набор результатов $D_i$, которые поисковая система представляет пользователю в ответ на его поисковый запрос. Результаты включают ссылки на различные документы $\{d_{i,1}, \dots, d_{i, M}\}$, которые поисковая система считает релевантными запросу пользователя.

Страницу результатов поиска, которую пользователь видит после ввода поискового запроса, принято называть серп, от SERP -- Search Engine Results Page. Серп обычно включает органические результаты (неоплаченные ссылки), платные результаты (рекламные ссылки), а также другие элементы, такие как изображения, видео, карты и фрагменты с ответами на вопросы.

\subsection{Пользовательская сессия}

Пользовательская сессия $S$ в контексте информационного поиска представляет собой последовательность взаимодействий пользователя с поисковой системой в течение определенного периода времени. В зависимости от контекста и целей анализа, можно выделить два типа сессий: сессии в широком смысле и сессии в узком смысле.

Сессия в широком смысле включает все действия пользователя, связанные с поиском информации, начиная с ввода первого поискового запроса $q_1$ и заканчивая завершением активности или истечением времени неактивности. В эту сессию входят все последующие перезапросы, изменения поисковых фраз, составляющие последовательность запросов $Q_N = {q_1, \dots, q_N}$, а также клики $C = \{c_{i,j}\}$ на результаты поиска, переходы по ссылкам и возвращения на страницу поиска.

Сессия в узком смысле ограничивается взаимодействием пользователя в рамках одного конкретного поискового запроса $q$. В нее входят действия, связанные только с этим запросом.

Действия пользователя зависят от типа показанного контента и не ограничиваются только кликами. Каждый тип принято называть вертикалью поиска. Например, если поисковая система показала видео в серпе, то гораздо важнее знать сколько времени пользователь его смотрел или досмотрел ли до конца.

\subsection{Релевантность}
Релевантность обозначает степень соответствия между поисковым запросом пользователя и предоставленными поисковой системой результатами. Чем выше степень соответствия между запросом и результатами, тем более релевантными считаются эти результаты для пользователя.

Иными словами, релевантность показывает, насколько хорошо результат поиска отвечает на запрос пользователя и удовлетворяет его информационные потребности. Это не только соответствие ключевым словам в запросе и содержанию документа, но и более общие факторы, такие как тематика, контекст и цель запроса.

\subsection{Кликовая модель}
Кликовая модель в классическом понимании -- это статистическая модель, которая предсказывает вероятность того, что пользователь совершит клик по определенному результату в поисковой выдаче, основываясь на исторических данных о кликах пользователей.

В отличии от других областей, например рекомендательных технологий, в поиске ключевым элементом является запрос и пользовательские сессии агрегируются по нему.

\section{Традиционные методы построения кликовых моделей}
\subsection{CTR}


\chapter{Основная часть}
\section{Формальная постановка задачи}

\section{Данные}


\section{Методы}

\section{Эксперимент}
\subsection{Результаты}


\chapter{Заключение}

\begin{thebibliography}{00}

\end{thebibliography}

\end{document}
